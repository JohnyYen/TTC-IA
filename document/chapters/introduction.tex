\section*{Introducción}
\addcontentsline{toc}{chapter}{Introducción}



	
	
	La hotelería, como rubro económico, ha evolucionado desde sus orígenes en las posadas medievales hasta convertirse en un sector estratégico en la economía global, particularmente en destinos turísticos como Cuba; donde su desarrollo histórico refleja la adaptación a demandas cambiantes, desde la provisión básica de alojamiento hasta la personalización de servicios bajo estándares internacionales. En este contexto, la competitividad en la industria hotelera no solo depende de la calidad de la infraestructura, sino también de la capacidad para anticipar y satisfacer las expectativas intangibles del cliente, como la experiencia emocional o la percepción de valor (3, 8). Sin embargo, factores económicos como fluctuaciones monetarias, naturales como eventos climáticos extremos y los cambios en el comportamiento del consumidor; impulsados por la digitalización y la sostenibilidad, han intensificado los desafíos para mantener la rentabilidad y la reputación en este mercado tan saturado.
	
	En el caso concreto de Cuba, la hotelería enfrenta dificultades estructurales derivadas de limitaciones tecnológicas, escasez de recursos y una infraestructura envejecida, contrastando con fortalezas como la hospitalidad tradicional y la riqueza cultural de este destino. Estudios recientes destacan que, aunque existen esfuerzos por incorporar innovaciones en la gestión, persisten barreras como la rigidez de los modelos administrativos y la falta de integración con sistemas de inteligencia de mercado (3, 4). La dependencia de alianzas estratégicas con cadenas internacionales ha permitido cierta modernización, pero no resuelve problemas críticos, como la baja eficiencia en la respuesta a las necesidades emergentes de los huéspedes. El impacto de esta realidad se evidencia en como las tensiones se agravan en un entorno donde la percepción del cliente, mediatizada por plataformas digitales como TripAdvisor, define gran parte del éxito comercial.
	
	Detectar las variables que impactan en la experiencia del cliente representa un reto central para la dirección hotelera. Las reseñas en línea, aunque ricas en información cualitativa, suelen ser analizadas de manera superficial, limitando su utilidad para decisiones estratégicas. Los métodos tradicionales de evaluación, como encuestas estructuradas, omiten matices contextuales presentes en comentarios textuales, mientras que, por el contrario, técnicas avanzadas de minería de textos permiten sistematizar estas percepciones (5, 6). Ejemplo de esto son los modelos probabilísticos, que han demostrado eficacia en la identificación de patrones temáticos recurrentes, vinculándolos a dimensiones críticas de la cadena de valor hotelera (5). De igual forma resulta relevante integrar estas herramientas en flujos analíticos escalables a fin de ofrecer una solución para transformar datos no estructurados en indicadores operativos, facilitando así intervenciones proactivas.
	
	Con esta base, se hace evidente que la fragmentación entre la recolección de reseñas y su análisis sistemático impide que las organizaciones hoteleras identifiquen oportunidades de mejora con precisión y rapidez. A pesar de la disponibilidad de técnicas de procesamiento del lenguaje natural (PLN), su aplicación en contextos como el cubano es limitada, generando brechas en la correlación entre percepciones del cliente y ajustes operativos.
	
	Es por ello que el presente trabajo pretende abordar el siguiente problema: 
	¿Cómo pueden las técnicas de minería de textos optimizar la extracción de conocimiento desde reseñas en plataformas digitales para identificar áreas de intervención en la gestión hotelera?
	
