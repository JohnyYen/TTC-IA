\usepackage[utf8]{inputenc} % Codificación de caracteres
\usepackage[T1]{fontenc} % Mejora la compatibilidad con caracteres especiales
\usepackage[spanish]{babel} % Idioma del documento
\usepackage{graphicx} % Para incluir imágenes
\usepackage{geometry} % Para ajustar márgenes
\usepackage{setspace} % Para ajustar el interlineado
\usepackage{float} % Para controlar la posición de figuras y tablas
\usepackage{tabularx}
\usepackage{ragged2e}
\usepackage{lipsum} % Para texto de relleno (puedes eliminarlo)
\usepackage{fancyhdr}
\usepackage{array}
\usepackage{csquotes}
\usepackage[style=numeric-verb, backend=biber, sorting=none]{biblatex} % Bibliografía
\usepackage[hypertexnames=false]{hyperref} % Para enlaces clicables (siempre al final)
\usepackage{multirow}
% Fuente
\renewcommand*{\familydefault}{\sfdefault}
\usepackage{uarial} % Carga la fuente Arial (aproximación)
\renewcommand\familydefault{\sfdefault} % Cambia la fuente predeterminada a sans-serif (Arial)
\usepackage{listings}
\usepackage{xcolor}
\usepackage[T1]{fontenc}


\addbibresource{Bibliography.bib}

% Configuración de márgenes
\geometry{a4paper, margin=1in, headsep=1cm}

% Configuración de interlineado
\onehalfspacing % o \doublespacing para doble espacio

% Configuración de hipervínculos
\hypersetup{
	colorlinks=true,
	linkcolor=blue,
	filecolor=magenta,      
	urlcolor=cyan,
	citecolor=blue,
	pdftitle={Título del documento},
	pdfauthor={Autor},
	pdfsubject={Tema},
	pdfkeywords={palabra1, palabra2}
}

%Gráficos
\graphicspath{{./src/}}

%Encabezado
\pagestyle{fancy}
\fancyhf{} % Limpia los encabezados y pies de página

\fancyhead[L]{\leftmark} % Nombre del capítulo en la izquierda
%\fancyhead[R]{\university} % Universidad en el centro
\fancyhead[R]{\thepage} % Número de página en la derecha

% Línea horizontal bajo el encabezado
\renewcommand{\headrulewidth}{0.4pt}


